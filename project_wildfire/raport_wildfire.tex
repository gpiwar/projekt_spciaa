\documentclass[onecolumn,12pt]{article}

\usepackage[utf8]{inputenc}
\usepackage[T1]{fontenc}
\usepackage{polski}

\setlength{\voffset}{-0.75in}
\setlength{\headsep}{5pt}

\usepackage{hyperref}
\hypersetup{
    colorlinks=false, %set true if you want colored link
    linktoc=all,     %set to all if you want both sections and subsections linked
}

\begin{document}

% ----------Strona tytułowa------------
\title{Symulacje procesów ciągłych i algorytmy adaptacyjne\\
Projekt –Symulacja pożaru}
\author{Magdalena Królikowska, Gabriela Piwar, Sławomir Tenerowicz}
\date{\today}
\maketitle

% ----------Spis treści------------
\tableofcontents
\thispagestyle{empty}
\newpage

% ----------Raport------------
\section{Wprowadzenie}
pożary straszne tak tak, trzeba przewidywać tak tak

\section{Opis pożaru}
linki co na grupie były wysłane, 

\section{Opis rozwiązania}
Tutaj screeny kodu, co zmieniliśmy i jak

\subsection{wiatr}
kod, obrazek funkcji (gif na upel sie wrzuci)

\subsection{początkowa konfiguracja pożaru}
zmiana punktu rozpoczecia pożaru i promienia, kod z funckji init state

\subsection{mapa paliwa}
maska binarna, kod z pythona, kod w wildfire.hpp, porownanie zdjecia satelitarnego 

\section{Wyniki}
obrazki jak wyszlo

\section{Podsumowanie}
udalo sie? 

%\bibliography{}

\end{document}